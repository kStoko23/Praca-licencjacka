\chapter*{Podsumowanie}
\par Celem oraz hipotezami\parrefnote{par:hipotezy} tej pracy dyplomowej było opracowanie, implementacja i analiza modelu optymalizacyjnego przeznaczonego do efektywnego zarządzania zasobami w firmie z sektora IT. Jako cel modelu ustalono minimalizację kosztów wynagrodzeń pracowników zespołu projektowego w firmie oraz dodano ograniczenia, których model musiał przestrzegać. Danymi wejściowymi dla modelu oraz badania, był syntetycznie generowany zestaw danych o pracownikach, zawierających różne metryki. Następnie zostało przeprowadzone badanie mające udowodnić użyteczność i poprawność modelu oraz rozumowania prowadzącego do jego powstania. Przeprowadzone badanie, polegające na analizie statystycznej wyników optymalizacji, dowiodło o skuteczności modelu, a wyniki badania zostały omówione w sekcji \refnote{sec:analiza}. Skutecznie przeprowadzone badanie potweirdza hipotezy oraz odpowiada twierdząco na pytanie: czy jest możliwym stworzenie, zaimplementowanie i przeanalizowanie w języku programowania Python modelu matematycznego, który stworzy optymalny zespół przy zadanych ograniczeniach.

\section*{Omówienia przeprowadzonego badania}
\par Praktyczna część pracy została poświęcona na studium przypadku opartego o zadanie optymalizacyjne (sekcja \refnote{sec:studium}). Problemem optymalizacyjnym, było utworzenie optymalnego zespołu pracowników, którzy w określonym zakresie czasu wykonają dany projekt przy jak najniższym koszcie wynagrodzeń oraz będą posiadać wysoki poziom umiejętności.
\par Przedstawiony problem wymagał zestawu danych, na który można by przeprowadzić optymalizację. W celu symulacji rzeczywistych warunków rynkowych, wygenerowano syntetyczne losowe dane w oparciu o rozkład normalny, w sposób opisany w sekcji \refnote{sec:kod_generator}. 
\par Optymalizacja została przeprowadzona za pomocą skryptu \verb|optimizer.py| (sekcja \refnote{subsec:optimizer_implementacja}) na wcześniej wygenerowanym zestawie danych. Otrzymane z niej wyniki zostały następnie poddane analizie statystycznej przez skrypt \verb|analyzer.py|, a otrzymane rezultaty i implementacja skryptu zostały omówione w sekcji \refnote{sec:analiza}. 
\par Przeanalizowane zostały dane wejściowe (populacja pracowników) oraz optymalny zespół stworzony przez model. Analiza wyników potwierdziła skuteczność modelu optymalizacyjnego. Optymalnie wybrany zespół posiadał wysoki poziom umiejętności (wyższy od poziomu populacji, wykres \figrefnote{fig:hist_suma_optimal}) oraz pracownicy oczekują dużo niższych wynagrodzeń niż średni pracownik w populacji (wykres \figrefnote{fig:hist_wynagrodzenie_optimal}). 
\par Takie efekty były oczekiwane od modelu, który z sukcesem dowiódł swojego działania tworząc optymalny, tani w finansowaniu i dobrze wyszkolony zespół. Z wyników badań można również wyciągnąć kilka innych wniosków, które zostały omówione w następnej sekcji.

\section*{Główne wnioski z pracy}
\par Przeprowadzone badanie przede wszystkim dowodzi o skuteczności utworzonego modelu. Tak utworzony model, dostosowany do potrzeb, mógłby również zostać zastosowany w rzeczywistym scenariuszu. Odpowiednio przygotowane dane wejściowe jak i dobrze określona funkcja celu wraz z ograniczeniami, mogą znacząco przyczynić się do poprawy procesu zarządzania zespołami projektowymi i nie tylko. 

\par Z wyników analizy, menedżerowie mogą dowiedzieć się o wpływie jaki ogólny poziom umiejętności ma na oczekiwania finansowe pracownika. Pozwala to na lepsze przygotowanie polityki wynagrodzeń oraz pomaga w przewidywaniu, jakich stawek będą oczekiwać pracownicy po zdobyciu nowych, znaczących umiejętności.

\par Dodatkowo menedżerowie mogą zauważyć, że do optymalnego zespołu nie został wybrany żaden z pracowników, którego suma umiejętności była bliska maksimum (patrz sekcja \refnote{subsec:tabela_populacja}, paragraf o \parrefnote{itm:max}). Pokazuje to, że do utworzenia optymalnego zespołu ze sztywnymi ograniczeniami budżetowymi, lepiej wybierać pracowników o mniejszej ekspertyzie (nadal powyżej średniej populacji \refnote{fig:hist_suma_optimal}) oczekujących niższego wynagrodzenia.
