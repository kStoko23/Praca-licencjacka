\chapter*{Wprowadzenie}

\par W erze cyfrowej, gdzie technologia ewoluuje w zawrotnym tempie, każdy projekt IT staje się coraz bardziej złożony i wielowymiarowy. Zarządzanie zasobami ludzkimi w branży IT jest skomplikowane, podobnie jak kody, które programiści piszą każdego dnia. W tym dynamicznym kontekście pojawia się pytanie: jak efektywnie zarządzać takimi zespołami, aby osiągnąć sukces przy optymalnym użyciu zasobów? Odpowiedź na to pytanie nie jest prosta, ale znalezienie odpowiedzi za pomocą optymalizacji matematycznej oraz modelu optymalizacyjnego już tak.

\par Każdy projekt IT jest unikatową mozaiką zadań, technologii, oczekiwań i ludzi. Według Thomasa Keen'a \parencite{keen2003creating} "\textit{W obecnych czasach liderzy zespołów stają przed coraz to nowymi wyzwaniami ciągle zmieniającego się świata; Niestabilność rynków wymusza podejmowanie szybkich kluczowych decyzji; olbrzymie fuzje i przejęcia; wyrafinowanie klientów oraz ciągłe rotacje i niepokój wśród pracowników}". W tym labiryncie zmiennych, menedżerowie projektów stają przed trudnym zadaniem zbudowania zespołu, który nie tylko posiada odpowiednie umiejętności techniczne, ale także potrafi efektywnie współpracować, komunikować się i wprowadzać innowacje. Wyzwanie to staje się szczególnie istotne w czasach, gdy rynek IT jest nasycony, a konkurencja stale rośnie.

\par Celem tej pracy jest opracowanie oraz implementacje takiego modelu optymalizacyjnego, który będzie w stanie wybrać optymalny zespół pracowników, spełniając wszystkie założone wymagania. Poprzez zastosowanie solidnych podstaw teoretycznych oraz przeprowadzenie praktycznych eksperymentów, praca ta chce dopowiedzieć na kluczowe pytanie: jak matematyczna precyzja może przekształcić chaos zarządzania projektami w dobrze funkcjonujący mechanizm. Jak twierdzi Al-mosa i in. w ostatnich latach, modelowanie matematyczne pełniło kluczową rolę w wielu sektorach życia. Od informatyki, przez fizykę i chemię aż do genetyki. \parencite{almosa2023python}.

\par W obliczu ciągle zmieniającego się świata IT i rosnącej konkurencji, utworzenie dobrze dobranego zespołu jest kluczowe dla sukcesu. Wedle Sweem S. \parencite{sweem2009leveraging}, w firmach utrzymujących się na rynku ciągle zachodzą zmiany. Jest to stale powodowane strategiami zarządzania talentów. Tezę te potwierdzają również autorzy \parencite{rusilowati2024optimizing}. Wedle przeprowadzonego przez autorów badania, w obecnie szybko rozwijającym się świecie, planowanie zasobów ludzkich stało się kluczowe dla organizacji. Stwierdza również, że dla wielu firm optymalizacja procesów (z zakresu zrządzania zespołami) nadal stanowi wyzwanie. Można z tego wywnioskować, że często zachodząca rotacja talentów w firmie może utrudniać ciągłe wybieranie nowych zespołów i znajdywanie tzw. "złotego środka". Jak twierdzi Ashcraft \parencite{ashcraft2011ipd}, zespoły dostarczają właściwej wiedzy we właściwym czasie, pobudzają kreatywność i obniżają bariery między uczestnikami projektu. Dodatkowo, procesy zespołowe usprawniają podejmowanie decyzji, a wspólne wybory zwiększają poparcie dla wybranych strategii. Niezwykle ważnym jest więc optymalne utworzenie zespołu, który będzie stać na wysokości zadania dla zapewnienia sukcesu projektu, w odpowiednio krótkim czasie.

\par W kontekście modelu optymalizacyjnego zaprezentowanego w tej pracy, celem jest minimalizacja całkowitego kosztu zatrudnienia pracowników przy jednoczesnym zapewnieniu, że zespół posiada niezbędne umiejętności techniczne kluczowe dla ukończenia projektu. Według Aithala'a \parencite{aithal2016theory}, optymalizacja wydajności jest trudnym procesem dla firmy, który wymaga wydobycia najwyższej produktywności z pracowników, przy utrzymaniu ograniczeń organizacyjnych. Celem pracy jest ułatwienie tego procesu za pomocą modelu optymalizacyjnego. Wspomniane ograniczenia zostaną zawarte w modelu, a ten wybierze optymalny zespół, nie przekraczając ich.

\par Zastosowanie takiego modelu w organizacjach IT, może przynieść wymierne korzyści. Według \parencite{wei2022optimal}, optymalizacja zasobów ludzkich polega na umieszczeniu odpowiednich ludzi, w odpowiednim miejscu w celu maksymalizacji ich wartości. Zastosowanie proponowanego modelu optymalizacyjnego może pomóc menedżerom dokonać właśnie tego. Dodatkowo pozwoli na lepsze zrozumienie, jak różne kombinacje umiejętności i kompetencji wpływają na wydajność projektu, co w efekcie prowadzi do bardziej świadomego i efektywnego zarządzania zasobami ludzkimi. Zarządzanie zasobami ludzkimi (HR) jest właśnie tym według autora \parencite{albi2024innovative}. Autor stwierdza, że HR jest domeną skupiającą się szczególnie na interakcjach i funkcjach osób w organizacji. Dotyczy to również optymalnego wykorzystania tych osób do osiągnięcia optymalnego poziomu wydajności i skuteczności w osiąganiu celów firmy. 

\par Ponadto, cel pracy obejmuje analizę literatury przedmiotu oraz przeprowadzenie badań, które mają na celu potwierdzenie skuteczności zaproponowanego modelu. Istotnym aspektem tej analizy jest zidentyfikowanie najlepszych praktyk w zakresie optymalizacji zasobów ludzkich oraz zrozumienie czym jest produktywność jednostki i co na nią wpływa. Hermawan określa produktywność pracowników jako wynik pracy osoby w określonym czasie, na określonej ilości zadań i ze spełnioną ilością celów \parencite{hermawan2020optimizing}. Natomiast Syverson \parencite{syverson2011determines} określa produktywność jako różnice między wykorzystanymi zasobami, a uzyskanym produktem przez firmę.

\par Kolejnym celem jest opracowanie narzędzia, które będzie łatwe w implementacji i użytkowaniu przez menedżerów projektów lub dział kadr (HR) w firmach IT. Model optymalizacyjny zaprojektowany w tej pracy ma być wsparciem w codziennych decyzjach menedżerskich, umożliwiając szybkie i efektywne tworzenie zespołów projektowych dostosowanych do specyficznych wymagań projektów. Dział HR, jak twierdzi autor \parencite{sihombing2024optimizing}, ogrywa ważną rolę w optymalizacji wydajności współpracy w firmie oraz napędza sukcesy organizacji. Spowodowane jest to, jak twierdzi Larson E. \parencite{larson2014project}, "\textit{ Wzrost gospodarczy jest osiągany przez nowe miejsca pracy i przewagę konkurencyjną, które są efektem ciągłej innowacji, opracowywania nowych produktów i usług oraz poprawy zarówno produktywności, jak i jakości pracy. To jest świat zarządzania projektami.}".

\par Implementacja modelu za pomocą technik programowania liniowego za pomocą języka Python i bibliotek takich jak \verb|PuLP|, ma na celu pokazanie praktycznych aspektów jego zastosowania. Wedle Mitchelle'a \parencite{mitchell2009introduction} użycie biblioteki \verb|PuLP| do modelowania i rozwiązywanie problemów optymalizacyjnych może być bardzo wygodnym podejściem dla programisty Python'a. Użycie tego pakietu pozwala skupić się na samym modelowaniu, bez potrzeby wymyślania i implementowania algorytmu do jego rozwiązania. Solvery używane przez \verb|PuLP| to dojrzałe oprogramowania, którym można powierzyć wybór odpowiednio algorytmu do optymalizacji. 



\par Podsumowując, celem niniejszej pracy jest nie tylko teoretyczne opracowanie modelu optymalizacyjnego, ale również kwestia jego praktycznego zastosowania w rzeczywistych scenariuszach zarządzania zespołami. Dostarczenie narzędzia, które wspiera menedżerów w podejmowaniu bardziej świadomych i efektywnych decyzji przy tworzeniu zespołów jest ważnym punktem tej pracy. 

