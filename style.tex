\usepackage[utf8]{inputenc}                                      
\usepackage[T1]{fontenc}  
\usepackage{helvet}
\renewcommand{\familydefault}{\sfdefault}
\usepackage{amsmath,amsfonts,amssymb,amsthm}
\usepackage{csquotes}
\usepackage[margin=2.5cm]{geometry}
\usepackage{graphicx}
\usepackage{xcolor}
\usepackage{array}
\graphicspath{ {images/} } % folder z grafiką
\usepackage{url}
\usepackage{hyperref}
\usepackage{indentfirst}
% % 3 kwietnia
\usepackage[american, polish]{babel}
\usepackage[backend=biber, giveninits=false, style=apa, sorting=nyt, autolang=other, language=american, defernumbers=true]{biblatex}
% % 3 kwietnia
\usepackage{listings}
\usepackage{fancyhdr}
\usepackage{etoolbox}
\usepackage{float}
\usepackage{cprotect}
\usepackage[nottoc, notlof, notlot, numbib]{tocbibind}
\usepackage{caption}


%\patchcmd{\chapter}{\thispagestyle{plain}}{\thispagestyle{fancy}}{}{}
% color def
\usepackage{color}
\definecolor{darkred}{rgb}{0.6,0.0,0.0}
\definecolor{darkgreen}{rgb}{0,0.50,0}
\definecolor{lightblue}{rgb}{0.0,0.42,0.91}
\definecolor{orange}{rgb}{0.99,0.48,0.13}
\definecolor{grass}{rgb}{0.18,0.80,0.18}
\definecolor{pink}{rgb}{0.97,0.15,0.45}


% General Setting of listings
\lstset{
  aboveskip=1em,
  breaklines=true,
  abovecaptionskip=-6pt,
  captionpos=b,
  escapeinside={\%*}{*)},
  frame=single,
  numbers=left,
  numbersep=15pt,
  numberstyle=\tiny,
  showstringspaces=false
}
% 0. Basic Color Theme
\lstdefinestyle{colored}{ %
  basicstyle=\ttfamily,
  backgroundcolor=\color{white},
  commentstyle=\color{green}\itshape,
  keywordstyle=\color{blue}\bfseries\itshape,
  stringstyle=\color{red},
}
% 1. General Python Keywords List
\lstdefinelanguage{PythonPlus}[]{Python}{
  morekeywords=[1]{,as,assert,nonlocal,with,yield,self,True,False,None,} % Python builtin
  morekeywords=[2]{,__init__,__add__,__mul__,__div__,__sub__,__call__,__getitem__,__setitem__,__eq__,__ne__,__nonzero__,__rmul__,__radd__,__repr__,__str__,__get__,__truediv__,__pow__,__name__,__future__,__all__,}, % magic methods
  morekeywords=[3]{,object,type,isinstance,copy,deepcopy,zip,enumerate,reversed,list,set,len,dict,tuple,range,xrange,append,execfile,real,imag,reduce,str,repr,}, % common functions
  morekeywords=[4]{,Exception,NameError,IndexError,SyntaxError,TypeError,ValueError,OverflowError,ZeroDivisionError,}, % errors
  morekeywords=[5]{,ode,fsolve,sqrt,exp,sin,cos,arctan,arctan2,arccos,pi, array,norm,solve,dot,arange,isscalar,max,sum,flatten,shape,reshape,find,any,all,abs,plot,linspace,legend,quad,polyval,polyfit,hstack,concatenate,vstack,column_stack,empty,zeros,ones,rand,vander,grid,pcolor,eig,eigs,eigvals,svd,qr,tan,det,logspace,roll,min,mean,cumsum,cumprod,diff,vectorize,lstsq,cla,eye,xlabel,ylabel,squeeze,}, % numpy / math
}
% 2. New Language based on Python
\lstdefinelanguage{PyBrIM}[]{PythonPlus}{
  emph={d,E,a,Fc28,Fy,Fu,D,des,supplier,Material,Rectangle,PyElmt},
}
% 3. Extended theme
\lstdefinestyle{colorEX}{
  basicstyle=\ttfamily,
  backgroundcolor=\color{white},
  commentstyle=\color{darkgreen}\slshape,
  keywordstyle=\color{blue}\bfseries\itshape,
  keywordstyle=[2]\color{blue}\bfseries,
  keywordstyle=[3]\color{grass},
  keywordstyle=[4]\color{red},
  keywordstyle=[5]\color{orange},
  stringstyle=\color{darkred},
  emphstyle=\color{pink}\underbar,
}

\lstset{style=colorEX}
\lstset{inputpath=chapters/Code/}


% % 3 kwietnia
\DefineBibliographyStrings{american}{
  andothers = {i in., \addabbrvspace}, % Zmienia "et al." na "i in."
}
\setlength{\emergencystretch}{3em}  % allows for up to 3em of additional white space per line
\tolerance=1000 
% % 3 kwietnia

\renewcommand{\lstlistingname}{Próbka kodu}
\renewcommand{\lstlistlistingname}{Spis próbek kodu}
\renewcommand{\listfigurename}{Spis rysunków}
\renewcommand{\listtablename}{Spis tabel}
\renewcommand{\figurename}{Rysunek}
\renewcommand{\tablename}{Tabela}

\newenvironment{conditions}
  {\par\vspace{\abovedisplayskip}\noindent\begin{tabular}{>{$}l<{$} @{${}={}$} l}}
  {\end{tabular}\par\vspace{\belowdisplayskip}}

\makeatletter
\let\orgdescriptionlabel\descriptionlabel
\renewcommand*{\descriptionlabel}[1]{%
\let\orglabel\label
\let\label\@gobble
\phantomsection
\edef\@currentlabel{#1}%
%\edef\@currentlabelname{#1}%
\let\label\orglabel
\orgdescriptionlabel{#1}%
}
\makeatother

\DeclareFieldFormat{labelnumberwidth}{#1\adddot\hspace{2mm}}
\setlength{\biblabelsep}{\labelsep}

% Dodanie numeracji do każdego wpisu w bibliografii z dodatkową spacją
\DeclareFieldFormat{labelnumber}{#1\adddot}
\renewbibmacro*{begentry}{%
  \printfield{labelnumber}%
  \hspace{2mm}% Dodatkowa przestrzeń po numerze
}

% Definicje stylu dla podpisów
\captionsetup[figure]{
  font=small, % Wielkość czcionki dla podpisów
  labelfont=bf, % Pogrubione oznaczenie numeru rysunku
  justification=raggedright, % Wyrównanie tekstu podpisu do lewej
  labelsep=period, % Dodanie kropki po numerze rysunku
  singlelinecheck=off % Wymuszenie wyrównania do lewej niezależnie od długości tekstu
}

\captionsetup[lstlisting]{
  font=small, % Wielkość czcionki dla podpisów
  labelfont=bf, % Pogrubione oznaczenie numeru rysunku
  justification=raggedright, % Wyrównanie tekstu podpisu do lewej
  labelsep=period, % Dodanie kropki po numerze rysunku
  singlelinecheck=off % Wymuszenie wyrównania do lewej niezależnie od długości tekstu
}


% Ustawienia dotyczące samych rysunków
\makeatletter
\renewcommand{\fnum@figure}{Rysunek \thefigure}
\makeatother

\setlength{\abovecaptionskip}{10pt} % Odstęp nad podpisem
\setlength{\belowcaptionskip}{5pt} % Odstęp pod podpisem
\newcommand{\refnote}[1]{%
  \ref{#1}\footnote{Sekcja \ref{#1}, strona \pageref{#1}}%
}
\newcommand{\parrefnote}[1]{%
  \ref{#1}\footnote{Paragraf \ref{#1}, strona \pageref{#1}}%
}
\newcommand{\figrefnote}[1]{%
  \ref{#1}\footnote{Rysunek \ref{#1}, strona \pageref{#1}}%
}
\newcommand{\pagerefnote}[1]{%
  \pageref{#1}\footnote{Strona \pageref{#1}}%
}